\newglossaryentry{sistemos_reikalavimas}{
  name={sistemos reikalavimas},
  description={
    tai sandoriu su užsakovu, specifikacija, standartu ar kitu juridinę
    galią turinčiu dokumentu numatyta sistemos savybė
    },
  sort={sistemos reikalavimas},
  plural={sistemos reikalavimai}
  }

\newglossaryentry{reikalavimu_inzinerija}{
  name={reikalavimų inžinerija},
  description={
    (pilnas pavadinimas – programų sistemų reikalavimų inžinerija)
    tai programų sistemų inžinerijos šaka, nagrinėjanti programų sistemų
    reikalavimų \emph{formulavimo}, \emph{specifikavimo}, \emph{analizės}
    ir \emph{vertinimo} klausimus
    }
  }

\newglossaryentry{operacinis_poreikis}{
  name={operacinis poreikis},
  description={
    tai skaičiavimo arba informacinė paslauga (dalykinė programa,
    duomenų bazė, interneto svetainė), padedanti naudotojams pasiekti
    numatytus operacinius tikslus (\gls{operacinis_tikslas}). Operacinius 
    poreikius galima traktuoti, kaip pačius bendriausius programų sistemos 
    reikalavimus (jie dar gali būti vadinami naudotojo reikalavimais)
    }
  }

\newglossaryentry{operacinis_tikslas}{
  name={operacinis tikslas},
  description={
    TODO: Parašyti.
    },
  plural={operaciniai tikslai}
  }

\newglossaryentry{operacinis_reikalavimas}{
  name={operacinis reikalavimas},
  description={
    \gls{operacinis_poreikis}, kuris yra išreikštas, kaip \gls{reikalavimas}
    (dažniausiai reikalaujama, kad jis būtų suprantamas tiek užsakovams, 
    tiek kuriantiems inžinieriams)
    },
  plural={operaciniai reikalavimai}
  }

\newglossaryentry{reikalavimas_programu_sistemai}{
  name={reikalavimas programų sistemai},
  description={
    TODO: Parašyti. (Nepamiršti paminėti, kuo skiriasi nuo operacinių
    reikalavimų.)
    (kaip ir operaciniai reikalavimai, taip ir programų sistemos 
    reikalavimai turi būti suprantami tiek užsakovams, tiek sistemą
    kuriantiems inžinieriams)
    },
  plural={reikalavimai programų sistemai}
  }

\newglossaryentry{reikalavimas}{
  name={reikalavimas},
  description={
    TODO: Parašyti.
    },
  plural={reikalavimai}
  }

\newglossaryentry{strategija}{
  name={strategija},
  description={
    tikslingi sprendimai ir veiksmai, kuriais, panaudojant verslo 
    kompetencijas ir resursus, realizuojamos neišnaudotos verslo galimybės
    ir šitaip pašalinamos jo problemos ir grėsmės
    }
  }
