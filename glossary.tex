\newglossaryentry{sistemos_reikalavimas}{
  name={sistemos reikalavimas},
  description={
    tai sandoriu su užsakovu, specifikacija, standartu ar kitu juridinę
    galią turinčiu dokumentu numatyta sistemos savybė
    },
  sort={sistemos reikalavimas},
  plural={sistemos reikalavimai}
  }

\newglossaryentry{reikalavimu_inzinerija}{
  name={reikalavimų inžinerija},
  description={
    (pilnas pavadinimas – programų sistemų reikalavimų inžinerija)
    tai programų sistemų inžinerijos šaka, nagrinėjanti programų sistemų
    reikalavimų \emph{formulavimo}, \emph{specifikavimo}, \emph{analizės}
    ir \emph{vertinimo} klausimus
    }
  }

\newglossaryentry{operacinis_poreikis}{
  name={operacinis poreikis},
  description={
    tai skaičiavimo arba informacinė paslauga (dalykinė programa,
    duomenų bazė, interneto svetainė), padedanti naudotojams pasiekti
    numatytus operacinius tikslus (\gls{operacinis_tikslas}). Operacinius 
    poreikius galima traktuoti, kaip pačius bendriausius programų sistemos 
    reikalavimus (jie dar gali būti vadinami naudotojo reikalavimais)
    }
  }

\newglossaryentry{operacinis_reikalavimas}{
  name={operacinis reikalavimas},
  description={
    \gls{operacinis_poreikis}, kuris yra išreikštas, kaip \gls{reikalavimas}
    (dažniausiai reikalaujama, kad jis būtų suprantamas tiek užsakovams, 
    tiek kuriantiems inžinieriams)
    },
  plural={operaciniai reikalavimai}
  }

\newglossaryentry{reikalavimas_programu_sistemai}{
  name={reikalavimas programų sistemai},
  description={
    TODO: Parašyti. (Nepamiršti paminėti, kuo skiriasi nuo operacinių
    reikalavimų.)
    (kaip ir operaciniai reikalavimai, taip ir programų sistemos 
    reikalavimai turi būti suprantami tiek užsakovams, tiek sistemą
    kuriantiems inžinieriams)
    },
  plural={reikalavimai programų sistemai}
  }

\newglossaryentry{reikalavimas}{
  name={reikalavimas},
  description={
    TODO: Parašyti.
    },
  plural={reikalavimai}
  }

\newglossaryentry{strategija}{
  name={strategija},
  description={
    tikslingi sprendimai ir veiksmai, kuriais, panaudojant verslo 
    kompetencijas ir resursus, realizuojamos neišnaudotos verslo galimybės
    ir šitaip pašalinamos jo problemos ir grėsmės
    }
  }

\newglossaryentry{tikslu_medis}{
  name={tikslų medis},
  description={
    \gls{semantinis_medis}, kurio viršūnės vaizduoja įvairių lygių tikslus,
    o lankai jungia tikslus su jų potiksliais (terminaliniai tikslai
    yra operacinio lygmens (\gls{operacinis_tikslas}) verslo tikslai)
    }
  }

\newglossaryentry{operacinis_tikslas}{
  name={operacinis tikslas},
  description={
    tikslas, kuris yra konkretus, pamatuojamas, duodantis konkretų 
    rezultatą ir apribotas laike
    },
  plural={operaciniai tikslai}
  }

\newglossaryentry{semantinis_medis}{
  name={semantinis medis},
  description={
    TODO: Paaiškinti.
    }
  }

\newglossaryentry{modelis}{
  name={modelis},
  description={
    aiškiai nusakytą tikslinę paskirtį turintis supaprastintas sistemos,
    proceso, reiškinio ar kokio nors kito originalo analogas, tapatus
    tam originalui modeliavimo tikslų požiūriu.
    }
  }

\newglossaryentry{abstraktusis_modelis}{
  name={abstraktusis modelis},
  description={
    teisingų tvirtinimų (teoremų) ir galbūt teiginių apie originalo
    statines ir dinamines charakteristikas rinkinys
    }
  }

\newglossaryentry{modeliavimo_kalbu_semantika}{
  name={modeliavimo kalbų semantika},
  description={
    % FIXME Suformuluoti normaliai.
    kokius teiginius galima užrašyti modeliavimo kalba
    }
  }

\newglossaryentry{modeliavimo_kalbu_notacija}{
  name={modeliavimo kalbų notacija},
  description={
    formalizmas, kuriuo užrašomi modelio teiginiai
    }
  }

\newglossaryentry{UML}{
  name={UML},
  description={
    standartinė grafinė kalba, pritaikyta specifikuoti, vizualizuoti,
    projektuoti, konstruoti ir dokumentuoti artefaktus (\gls{artefaktas}),
    sukuriamus, kuriant ne tik programų, bet ir verslo bei kitokio
    pobūdžio sistemas (pilnas pavadinimas – \emph{Unified Modeling 
    Language})
    }
  }

\newglossaryentry{ansamblis}{
  name={ansamblis},
  description={
    tarpusavio susiderinimas, darni visuma (DLKŽ)
    }
  }

\newglossaryentry{artefaktas}{
  name={artefaktas},
  description={
  TODO: 7 temos 704 skaidrė.
    }
  }
