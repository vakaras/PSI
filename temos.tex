\chapter{Temų sąrašas}

\begin{enumerate}
  \item \emph{3 tema} Įvadas į kursą (2 skaidrė)
    \begin{itemize}
      \item 8 dalis. Menas, mokslas ar inžinerija? (8 skaidrė)
      \item 9 dalis. Sistemos ir sistemų inžinerija. (63 skaidrė)
      \item 10 dalis. Įmonių inžinerija (102 skaidrė)
      \item Informacinė sistema (132 skaidrė)
      \item Programinė įranga (147 skaidrė)
      \item Programų sistema (178 skaidrė)
      \item Įmonių inžinerija (252 skaidrė)
      \item \emph{4 tema} 11 dalis. Chroniška programavimo krizė (5 skaidrė)
      \item 12 dalis. Pramoniniai programų sistemų kūrimo metodai (96 
        skaidrė)
      \item Darbo organizavimas (132 skaidrė)
      \item Projektavimas (186 skaidrė)
      \item Darbo našumas (198 skaidrė)
      \item Masinė gamyba (277 skaidrė)
      \item Standartizavimas (285 skaidrė)
      \item Dokumentavimas (303 skaidrė)
      \item Sandoris (319 skaidrė)
      \item Proceso inžinerija (328 skaidrė)
      \item Projekto valdymas (336 skaidrė)
      \item Kokybės užtikrinimas (374 skaidrė)
      \item Komponentai (385 skaidrė)
      \item 13 dalis. PS inžinerija nėra jokia stebuklingoji lazdelė
        (404 skaidrė)
      \item \emph{5 tema} 14 dalis. Ką apima PS inžinerija? (5 skaidrė)
      \item Procesai ir artifaktai (50 skaidrė)
      \item 15 dalis. Principai ir paradigmos. (62 skaidrė)
    \end{itemize}
\end{enumerate}

Reikalavimų inžinerija:

\begin{enumerate}
  \item \emph{6 tema} Įvadas (5 skaidrė)
  \item 1 dalis. Išorinė analizė (11 skaidrė)
  \item 1.1 dalis. Poreikių analizė (12 skaidrė)
\end{enumerate}

Reikalavimų inžinerija (2 dalis. Koncepcinis modeliavimas):

\begin{enumerate}
  \item \emph{7 tema} Modeliai (13 skaidrė)
  \item UML (44 skaidrė)
    \begin{itemize}
      \item Kas yra UML? (45 skaidrė)
      \item Užduočių diagramos (? skaidrė)
      \item Sekų diagramos (200 skaidrė)
      \item Komunikavimo diagramos (249 skaidrė)
      \item Klasių diagramos (268 skaidrė)
      \item Objektų diagramos (451 skaidrė)
      \item Būsenų diagramos (464 skaidrė)
      \item Užduočių, sekų, klasių ir būsenų diagramų susiejimas (574 
      \item skaidrė)
      \item Veiklos diagramos (575 skaidrė)
      \item Komponentų diagramos (658 skaidrė)
      \item Išdėstymo diagramos (734 skaidrė)
      \item Paketų diagramos (781 - 816 skaidrės)
    \end{itemize}
  \item \emph{8 tema} Užduočių modeliavimas.
\end{enumerate}

Reikalavimų  inžinerija (3 dalis. Reikalavimai):

\begin{enumerate}
  \item \emph{9 tema} Kas yra reikalavimas?
  \item ISO 9126, Quint (86 skaidrė)
  \item Anotuoti reikalavimai (142 skaidrė)
  \item Reikalavimų specifikacija (147 skaidrė)
\end{enumerate}

Reikalavimų  inžinerija (4 dalis. Dalykinės srities analizė):

\emph{10 tema}

Projektavimas ir testavimas.

\emph{11 tema}
