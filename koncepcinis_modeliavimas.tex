\section{Koncepcinis modeliavimas}

\subsection{Modeliai}

\begin{itemize}
  \item Modelio samprata.
  \item Abstraktieji modeliai.
  \item Modelių paskirtis.
  \item Modelių lygmenys.
  \item Semantika ir pateiktis.
  \item Kontekstas.
\end{itemize}

\subsection{UML}

\subsubsection{Kas tai yra UML?}

\begin{itemize}
  \item UML istorija.
  \item UML diagramos.
\end{itemize}

\subsubsection{Užduočių diagramos}

\begin{itemize}
  \item Elementai.
  \item Notacija.
  \item Paskirtis.
  \item Diagramos patikslinimai.
  \item Kaip nustatyti, kokias užduotis modeliuoti.
  \item Tipinės modeliavimo klaidos.
  \item Kardinalumas ir agentų apibendrinimas.
  \item Užduotys.
  \item Agentai (aktoriai).
  \item Asociacijos ir priklausomybės («includes», «extends», 
    apibendrinimas).
  \item Laiko ir kitų trigerių modeliavimas.
  \item Užduočių ir funkcijų skirtumai.
\end{itemize}

\subsubsection{Paketai}

\begin{itemize}
  \item Rūšys.
  \item Matomumas.
  \item Prieigos kelias.
  \item Užduočių paketavimas.
  \item Pastabos.
  \item Priklausomybės ir ribojimai.
  \item Komentarai.
  \item Savybės.
  \item Verslo sistemos ir programų sistemos užduotys.
  \item Užduočių realizavimas.
\end{itemize}

\subsubsection{Sekų diagramos}

\begin{itemize}
  \item Elementai.
  \item Notacija.
  \item Paskirtis.
  \item Gyvavimo atkarpos.
  \item Objektai ir agentai (aktoriai).
  \item Pranešimai.
  \item Konstruktoriai ir destruktoriai.
  \item Lokalieji kintamieji.
  \item Fragmentai.
  \item Šakojimasis (alternatyvios scenarijaus šakos).
  \item Ciklai.
  \item Bendro naudojimo fragmentai.
  \item Dekompozicija.
  \item Sąveikos su išore taškai.
  \item Papildomos konstrukcijos.
\end{itemize}

\subsubsection{Komunikavimo diagramos}

\begin{itemize}
  \item Elementai.
  \item Notacija.
  \item Paskirtis.
  \item Ansambliai.
  \item Objektai komunikavimo diagramose.
  \item Pranešimų modeliavimas.
  \item Sąryšiai komunikavimo diagramose.
  \item Konstruktoriai ir destruktoriai.
  \item Navigavimas ansambliuose.
  \item Multiobjektai.
  \item Aktyvūs ir pasyvūs objektai.
  \item Tipinių projektavimo sprendimų modeliavimas.
\end{itemize}

\subsubsection{Klasių diagramos}

Tingiu… (7 tema, 7 skaidrė)

\subsubsection{Būsenų diagramos}

Tingiu… (7 tema, 8 skaidrė)

\subsubsection{Veiklos diagramos}

\begin{itemize}
  \item Elementai.
  \item Notacija.
  \item Paskirtis.
  \item Objekto atsakomybė.
  \item Veiksmo būsenos.
  \item Lankai veiklos diagramoje.
  \item Veiklos diagramos kaip apibendrinantis formalizmas.
  \item Valdymo srautas.
  \item Objektų srautas.
  \item Objekto būsena.
  \item Žymės.
  \item Veiklos ir veiksmai.
  \item Veiklos viršūnės.
  \item Prieš ir po sąlygos.
  \item Veiklų modeliavimas.
  \item Sprendimai.
  \item Iššakojimas.
  \item Atsakomybės juostos.
  \item Veiksmų ir objektų srautų modeliavimas.
  \item Įvykių inicijuojami veiksmai.
  \item UML diagramų tarpusavio ryšiai.
\end{itemize}

\subsubsection{Komponentų diagramos}

\begin{itemize}
  \item Elementai.
  \item Notacija.
  \item Paskirtis.
  \item Komponentai.
  \item Komponentų sąsajos.
  \item Kontraktiniai įsipareigojimai.
  \item Vidiniai klasifikatoriai.
  \item Išorinis komponento rodinys.
  \item Vidinis komponento rodinys.
  \item Artefaktai.
  \item Komponentų rūšių specifikacijos.
  \item Komponentų tipai ir egzemplioriai.
  \item Komponentų abstrakcijos lygmenys.
  \item Komponentų grafas.
  \item Komponentų ir paketų skirtumai.
  \item Komponentų ir klasių skirtumai.
  \item Tinklo mazgai.
  \item Komponentai ir užduotys.
\end{itemize}

\subsubsection{Išdėstymo diagramos}

\begin{itemize}
  \item Elementai.
  \item Notacija.
  \item Paskirtis.
  \item Komponentų ir išdėstymo diagramų skirtumai.
  \item Tipų lygmens diagramos.
  \item Išdėstymo mazgai.
  \item Komunikavimo maršrutai.
  \item Artefaktai.
  \item Išdėstymo specifikacija.
  \item Įrenginiai.
  \item Vykdymo aplinka.
  \item Migruojantys komponentai.
  \item Manifestavimas.
\end{itemize}

\subsubsection{Paketų diagramos}

\begin{itemize}
  \item Elementai.
  \item Notacija.
  \item Paskirtis.
  \item Prieigos keliai.
  \item Paketų tarpusavio priklausomybės.
  \item Paketų importavimas.
  \item Paketų suliejimas.
  \item Sistemos struktūros modeliavimas paketais.
  \item Modelių pakavimas.
\end{itemize}

\subsubsection{Papildomos priemonės}

\begin{itemize}
  \item Pastabos.
  \item Ribojimai.
  \item Komentarai.
  \item Žymėtosios reikšmės.
  \item Konstruktorių rūšys.
\end{itemize}
