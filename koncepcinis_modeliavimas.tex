\section{Koncepcinis modeliavimas}

\subsection{Modeliai}

\begin{itemize}
  \item Modelio samprata.
  \item Abstraktieji modeliai.
  \item Modelių paskirtis.
  \item Modelių lygmenys.
  \item Semantika ir pateiktis.
  \item Kontekstas.
\end{itemize}

% 7 temos 15 skaidrė.

\Gls{modelis} – \glsentrydesc{modelis}. Modelio naudojimo vietoj originalo
priežastis yra ta, kad juo lengviau naudotis siekiamų tikslų požiūriu.
Todėl jo forma turi būti tokia, kuria naudojantis yra patogiausia
pasiekti modeliavimo tikslus.

\Gls{abstraktusis_modelis} – \glsentrydesc{abstraktusis_modelis}.
Tvirtinimai ir teiginiai užrašomi kokia nors modeliavimo kalba 
(pavyzdžiui, UML). Modeliavimo kalbos gali būti įvairaus pobūdžio,
modeliuose gali būti leidžiama panaudoti piešinius, tekstą bei formules.
Nagrinėjant abstrakčiuosius modelius, galima nagrinėti modeliavimo
kalbų semantiką (\gls{modeliavimo_kalbu_semantika}) ir jų notaciją
(\gls{modeliavimo_kalbu_notacija}).

% FIXME Pertvarkyti: 7 temos 19 skaidrė.
Modelių panaudojimo tikslai programų sistemų inžinerijoje:
\begin{itemize}
  \item atskleisti ir tiksliai aprašyti verslo sistemų struktūrą ir
    elgseną (pavyzdžiui, verslas gali būti modeliuojamas, siekiant atlikti
    vidinę jo analizę);
  \item tiksliai aprašyti būsimos programų sistemos reikalavimus ir
    jos struktūrą bei elgseną (įvairiais abstrakcijos lygmenimis) ir
    tuo pačiu sudaryti prielaidas visiems projekto dalyviams tiksliai ir
    vienareikšmiškai susitarti kokia ta sistema turėtų būti 
    (reikalavimų modeliavimas, atliekant koncepcinį modeliavimą).
  \item apgalvoti programų sistemos projektą;
  \item aprašyti projektavimo sprendimus;
  \item generuoti įvairius tarpinius darbo rezultatus;
  \item organizuoti, ieškoti, filtruoti, analizuoti ir redaguoti
    informaciją apie dideles sistemas;
  \item ekonomiškai vertinti skirtingus projektavimo sprendimus;
  \item kuriamai sistemai įgyvendinti.
\end{itemize}

Programų sistemų inžinerijoje modeliai yra naudojami:
\begin{itemize}
  \item modeliuoti kompiuterizuojamas verslo sistemas;
  \item modeliuoti kuriamų programų sistemų reikalavimus;
  \item stambiu planu modeliuoti projektuojamas programų sistemas;
  \item dokumentuoti priimamus projektavimo sprendimus;
  \item generuoti įvairius tarpinius darbo rezultatus;
  \item organizuoti, ieškoti, filtruoti, analizuoti ir redaguoti 
    informaciją apie dideles programų sistemas;
  \item ekonomiškai vertinti alternatyvius projektavimo sprendimus;
  \item įgyvendinti kuriamas programų sistemas.
\end{itemize}

Svarbiausieji modeliavimo lygmenys:
\begin{enumerate}
  \item \emph{reikalavimų specifikavimas} – ankstyvosiose projekto 
    stadijose kuriami aukšto abstrakcijos lygmens modeliai padeda
    projekto dalyviams mąstyti kryptingai ir išryškina svarbiausias
    sistemos savybes, tokie modeliai aprašo sistemos reikalavimus ir
    yra naudojami kaip išeities taškas programų sistemų projektavime;
  \item \emph{programų sistemos struktūros specifikavimas stambiu planu}
    – pradinėse projektavimo stadijose kuriamuose modeliuose akcentuojami
    svarbiausieji būsimos programų sistemos struktūriniai ypatumai, jų
    paskirtis yra nesigilinant į detales, įsitikinti pagrindinių
    projektavimo sprendimų korektiškumu;
  \item \emph{detalus galutinės sistemos specifikavimas} – gaunama
    visa sistemos kūrimui reikalinga informacija (veikimo logika,
    algoritmai, duomenų struktūros, sistemos našumo užtikrinimo
    mechanizmai bei organizaciniai sprendimai apie tai, kaip bus
    dirbama su sistema);
  \item \emph{tipinių arba galimų sistemos naudojimo ar veikimo pavyzdžių
    pateiktis} – daugumai žmonių mąsto ne abstrakčiai, bet manipuliuodami
    pavyzdžiais, todėl tam, kad jie galėtų suprasti sistemą būtina
    parengti tinkamus pavyzdžius;
  \item \emph{išsamus arba dalinis sistemos aprašymas} – 
    % FIXME Išsiaiškinti kuo skiriasi nuo „detalaus galutinės sistemos
    % specifikavimo“. (7 temos 39 skaidrė)
\end{enumerate}

Programų sistemų modeliai turi du aspektus:
\begin{itemize}
  % TODO Suprasti ir išsiaiškinti. 7 temos 42 skaidrė.
  \item \emph{semantinį} – modelyje saugoma prasminė informacija, kuri
    yra išreikšta panaudojant tam tikrą notaciją (sintaksę);
  \item \emph{vizualinį} – modelis pateikiamas pavidalu, skirtu palengvinti
    supratimą.
\end{itemize}

Modeliai, patys savaime saugo tik dalį informacijos. Norint ja tinkamai
pasinaudoti reikia turėti kontekstą, kuris nurodo kaip turi būti 
interpretuojama modeliuose pateikta informacija ir kaip siejasi modelių
informacija tarpusavyje.

\subsection{UML}

% 7 temos 44 skaidrė

\subsubsection{Kas tai yra UML?}

\begin{itemize}
  \item UML istorija.
  \item UML diagramos.
\end{itemize}

\gls{UML} – \glsentrydesc{UML}. Ši kalba yra pagrįsta objektine paradigma.
Objektinėje paradigmoje sistema yra traktuojama, kaip rinkinys tarpusavyje
susietų ir pranešimais besikeičiančių objektų. Objektai sudaro ansamblį
(\gls{ansamblis}), jame yra pasiskirstę vaidmenimis ir visi kartu siekia
tikslų, numatytų tos sistemos reikalavimais.

Objektinėje paradigmoje objektai turi:
\begin{itemize}
  \item \emph{tapastis} – % FIXME Išsiaiškinti: kas tai? 7 temos 46 skaidrė
  \item \emph{būsenas} – jos nusakomos objekto savybių (atributų)
    reikšmėmis;
  \item \emph{elgseną} – ji aprašoma vykdomomis operacijomis, kurios
    yra vykdomos, kaip atsakas į objekto gautus pranešimus.
\end{itemize}

Modeliuojant sistemas UML yra naudojama trylika\footnote{UML 2 turi 
daugiau} skirtingų diagramų:
\begin{itemize}
  \item elgsenos diagramos:
    \begin{itemize}
      \item \emph{užduočių diagrama};
      \item sąveikos diagramos:
        \begin{itemize}
          \item \emph{sekų diagrama};
          \item \emph{komunikavimo diagrama};
        \end{itemize}
    \end{itemize}
  \item statinės struktūros diagramos:
    \begin{itemize}
      \item \emph{klasių diagrama};
      \item \emph{objektų diagrama};
      \item \emph{paketų diagrama};
    \end{itemize}
  \item dinaminio modeliavimo diagramos:
    \begin{itemize}
      \item \emph{būsenų diagrama};
      \item \emph{veiklos diagrama};
    \end{itemize}
  \item realizavimo diagramos:
    \begin{itemize}
      \item \emph{komponentų diagrama};
      \item \emph{išdėstymo diagrama}.
    \end{itemize}
\end{itemize}

% FIXME Kur šitai grūsti? 7 tema 62 skaidrė.
Sąveikos diagramos parodo objektų tarpusavio sąveiką sistemos vykdymo
metu. Sąveikos diagramos apima:
\begin{itemize}
  \item sekų diagramas, parodančias objektų sąveikos nuoseklumą;
  \item komunikavimo diagramas, parodančias sąveikoje dalyvaujančių objektų
    struktūrą;
  \item sąveikos apžvalgos diagramas, susiejančias sąveikos fragmentus su
    aukšto lygmens darbų srautais;
  \item chronometražo \eng{timing} diagramos, skirtos laiko ribojimams
    aprašyti modeliuojant realiame laike veikiančias sistemas.
\end{itemize}

% 7 temos 68 skaidrė.
UML diagramos naudojamos trims skirtingiems tikslams:
\begin{enumerate}
  \item \emph{koncepciniam modeliavimui} – modeliuoti realaus pasaulio
    sistemas (pavyzdžiui, verslo sistemas);
  \item \emph{kuriamų sistemų specifikavimui}, bei jų projektavimui
    koncepciniu ir architektūriniu lygmenimis;
  \item \emph{kuriamoms sistemoms realizuoti} – projektuoti eskiziniu
    ir detaliuoju lygmenimis.
\end{enumerate}

% TODO Suprasti ir įvesti 7 temos 72-78 skaidrė

\subsubsection{Užduočių diagramos}

\begin{itemize}
  \item Elementai.
  \item Notacija.
  \item Paskirtis.
  \item Diagramos patikslinimai.
  \item Kaip nustatyti, kokias užduotis modeliuoti.
  \item Tipinės modeliavimo klaidos.
  \item Kardinalumas ir agentų apibendrinimas.
  \item Užduotys.
  \item Agentai (aktoriai).
  \item Asociacijos ir priklausomybės («includes», «extends», 
    apibendrinimas).
  \item Laiko ir kitų trigerių modeliavimas.
  \item Užduočių ir funkcijų skirtumai.
\end{itemize}

% 7 temos 55 skaidrė.

Užduočių \eng{use case} diagrama parodo sistemos išorėje esančius sistemos 
naudotojus
ir jų sąryšius su sistemos vykdomomis užduotimis (sistemos panaudojimo
būdais). Užduočių diagramomis yra aprašomas verslo, programų ar kokios
nors kitos sistemos funkcionalumas. Dažniausiai užduočių esmė yra
aprašoma prie diagramos papildomai pridedamu formalizuotos struktūros
tekstiniu aprašu. Sistema yra traktuojama, kaip juodoji dėžė.

% 7 temos 79 skaidrė.
Užduočių diagramomis modeliuojamas globalus (stambaus plano) požiūris
į sistemą – paprastai jis vadinamas išoriniu požiūriu.

% TODO Suprasti ir įvesti: 7 temos 80-145 skaidrė.

% 7 temos 146 skaidrė.
Užduočių ir funkcijų skirtumai:
\begin{itemize}
  \item užduotys aprašo tam tikrus sistemos naudotojo tikslus;
  \item sistemos funkcijos realizuoja užduotis.
\end{itemize}

% TODO Suprasti ir įvesti: 7 temos 148-199 skaidrė.

\subsubsection{Paketai}

\begin{itemize}
  \item Rūšys.
  \item Matomumas.
  \item Prieigos kelias.
  \item Užduočių paketavimas.
  \item Pastabos.
  \item Priklausomybės ir ribojimai.
  \item Komentarai.
  \item Savybės.
  \item Verslo sistemos ir programų sistemos užduotys.
  \item Užduočių realizavimas.
\end{itemize}

\subsubsection{Sekų diagramos}

\begin{itemize}
  \item Elementai.
  \item Notacija.
  \item Paskirtis.
  \item Gyvavimo atkarpos.
  \item Objektai ir agentai (aktoriai).
  \item Pranešimai.
  \item Konstruktoriai ir destruktoriai.
  \item Lokalieji kintamieji.
  \item Fragmentai.
  \item Šakojimasis (alternatyvios scenarijaus šakos).
  \item Ciklai.
  \item Bendro naudojimo fragmentai.
  \item Dekompozicija.
  \item Sąveikos su išore taškai.
  \item Papildomos konstrukcijos.
\end{itemize}

% 7 tema 63 skaidrė
Sekų diagrama visuomet yra susiejama su konkrečia užduotimi ir parodo, kaip
objektai, keisdamiesi pranešimais, įgyvendina ta užduotimi numatytą
funkcionalumą.

% 7 tema 200 skaidrė.

% TODO: Suprasti ir įvesti 201-248.

\subsubsection{Komunikavimo diagramos}

\begin{itemize}
  \item Elementai.
  \item Notacija.
  \item Paskirtis.
  \item Ansambliai.
  \item Objektai komunikavimo diagramose.
  \item Pranešimų modeliavimas.
  \item Sąryšiai komunikavimo diagramose.
  \item Konstruktoriai ir destruktoriai.
  \item Navigavimas ansambliuose.
  \item Multiobjektai.
  \item Aktyvūs ir pasyvūs objektai.
  \item Tipinių projektavimo sprendimų modeliavimas.
\end{itemize}

% 7 temos 64 skaidrė.

Komunikavimo diagrama parodo elementų vaidmenis, atsakomybes ir ryšius
konkrečios užduoties, objekto, ansamblio, klasės ar veiklos kontekste.

% 7 tema 249 skaidrė.

% TODO: Suprasti ir įvesti 250-267

\subsubsection{Klasių diagramos}

Tingiu… (7 tema, 7 skaidrė)

% 7 temos 56 skaidrė.

Klasių diagrama aprašo statinę sistemos struktūrą. Klasės aprašo tai,
ką apdoroja sistema (verso sistemoje – verslo objektus, informacinėje
sistemoje – informacinius objektus, programų sistemoje – duomenis).
Jos gali būti susijusios tarpusavyje įvairiais būdais:
\begin{itemize}
  \item asociacija – susietos ryšiais;
  \item priklausymu tam pačiam paketui;
  \item specializacija – viena klasė yra kitos klasės specializacija;
  \item priklausymu – pavyzdžiui, viena klasė gali naudoti kitą.
\end{itemize}
Klasių diagramose parodoma vidinė klasių struktūra ir visi klasių
tarpusavio ryšiai.

% 7 temos 58 skaidrė.
\emph{Objektų diagramos} yra atskiras klasių diagramų atvejis. Vietoj
klasių, jose vaizduojami tų klasių objektai. Objektų diagrama yra konkreti
klasių diagramos realizacija pasirinktu laiko momentu. Ji naudojama
klasių diagramai iliustruoti.

% 7 tema 268 skaidrė.

% TODO: Suprasti ir įvesti 269-321

Klasės ir tipai:
\begin{itemize}
  \item \emph{tipas}:
    \begin{itemize}
      \item protokolas, kurį supranta objektas;
      \item leistinų operacijų rinkinys;
    \end{itemize}
  \item \emph{klasė}:
    \begin{itemize}
      \item realizavimo konstrukcija;
      \item realizuoja vieną ar kelis tipus.
    \end{itemize}
\end{itemize}
Java programavimo kalboje tipai aprašomi sąsajomis, o C++ – abstrakčiomis
klasėmis.

% TODO: Suprasti ir įvesti 7 temos 322-358.
reifikacija – filosofijoje sudaiktinimas, laikymas ko nors daiktu, nors
tai nėra daiktas (DLKŽ).

% TODO: Suprasti ir įvesti 7 temos 360-450.
% Objektų diagramos 7 temos 451 skaidrė
% TODO: Suprasti ir įvesti 7 temos 452-463

\subsubsection{Būsenų diagramos}

Tingiu… (7 tema, 8 skaidrė)

% 7 temos 59 skaidrė.
Būsenų diagramos (mašinos) papildo klasių diagramą. Kiekviena klasė gali
turėti savą būsenų diagramą, parodančią visas galimas tos klasės objektų
būsenas. Būsenų diagramos gali būti dviejų rūšių:
\begin{itemize}
  % FIXME Suprasti ir paaiškinti. 7 temos 60 skaidrė.
  \item \emph{elgseną aprašančios būsenų mašinos} – jose parodoma, kokie
    išoriniai įvykiai iššaukia būsenų pokyčius ir kokie veiksmai yra
    atliekami keičiant būsenas (kitaip tariant, yra aprašomi klasių
    gyvavimo ciklai);
  \item \emph{protokolus aprašančios būsenų mašinos} – jose pagrindinis
    dėmesys yra skiriamas perėjimams iš būsenos į būseną ir taisyklėms,
    nusakančioms operacijų vykdymo tvarką.
\end{itemize}
Ji taip pat parodo, kokie išoriniai įvykiai iššaukia būsenų
pokyčius ir kokie veiksmai yra atliekami keičiant būsenas. Tokios
diagramos vadinamos elgseną aprašančiomis būsenų mašinomis. Be jų dar yra
protokolus aprašančios būsenų mašinos.

% 7 temos 464 skaidrė.

% TODO Suprasti ir įdėti 7 temos 465-574. (Visiškai nepagavau.)

\subsubsection{Veiklos diagramos}

\begin{itemize}
  \item Elementai.
  \item Notacija.
  \item Paskirtis.
  \item Objekto atsakomybė.
  \item Veiksmo būsenos.
  \item Lankai veiklos diagramoje.
  \item Veiklos diagramos kaip apibendrinantis formalizmas.
  \item Valdymo srautas.
  \item Objektų srautas.
  \item Objekto būsena.
  \item Žymės.
  \item Veiklos ir veiksmai.
  \item Veiklos viršūnės.
  \item Prieš ir po sąlygos.
  \item Veiklų modeliavimas.
  \item Sprendimai.
  \item Iššakojimas.
  \item Atsakomybės juostos.
  \item Veiksmų ir objektų srautų modeliavimas.
  \item Įvykių inicijuojami veiksmai.
  \item UML diagramų tarpusavio ryšiai.
\end{itemize}

% 7 temos 61 skaidrė
Veiklos aprašo veiksmų srautus. Pagrindinė jų paskirtis yra aprašyti
procesų darbų srautus, tačiau jos gali būti panaudotos ir kitiems tikslams.
Pavyzdžiui, užduočių ar procedūrų valdymo srautams aprašyti.

\begin{note}
  Veiklos diagramoje galima aprašyti ne tik tai, kokie veiksmai ir kokia
  eilės tvarka yra vykdomi atliekant veiklą, bet ir siunčiamus pranešimus
  bei objektus, perduodamus iš vieno veiksmo kitam.
\end{note}

% 7 temos 575 skaidrė.

% TODO Išsiaiškinti kas per velnias yra „klasifikatorius“ (7 temos 581).
% TODO Išsiaiškinti ir įdėti: 7 temos 576-657.

\subsubsection{Komponentų diagramos}

\begin{itemize}
  \item Elementai.
  \item Notacija.
  \item Paskirtis.
  \item Komponentai.
  \item Komponentų sąsajos.
  \item Kontraktiniai įsipareigojimai.
  \item Vidiniai klasifikatoriai.
  \item Išorinis komponento rodinys.
  \item Vidinis komponento rodinys.
  \item Artefaktai.
  \item Komponentų rūšių specifikacijos.
  \item Komponentų tipai ir egzemplioriai.
  \item Komponentų abstrakcijos lygmenys.
  \item Komponentų grafas.
  \item Komponentų ir paketų skirtumai.
  \item Komponentų ir klasių skirtumai.
  \item Tinklo mazgai.
  \item Komponentai ir užduotys.
\end{itemize}

% 7 temos 65 skaidrė.
% FIXME Kaip suprasti „fizinę struktūrą“?
Komponentų diagrama aprašo fizinę sistemos struktūrą (iš kokių komponentų
yra sudaryta ta sistema). Diagramoje kiekvienam komponentui yra parodoma, 
kokias klases tas komponentas įgyvendina bei parodomos visos komponentų
tarpusavio priklausomybės.

% 7 temos 658 skaidrė.

% TODO Išsiaiškinti ir įdėti: 7 temos 659-734.

\subsubsection{Išdėstymo diagramos}

\begin{itemize}
  \item Elementai.
  \item Notacija.
  \item Paskirtis.
  \item Komponentų ir išdėstymo diagramų skirtumai.
  \item Tipų lygmens diagramos.
  \item Išdėstymo mazgai.
  \item Komunikavimo maršrutai.
  \item Artefaktai.
  \item Išdėstymo specifikacija.
  \item Įrenginiai.
  \item Vykdymo aplinka.
  \item Migruojantys komponentai.
  \item Manifestavimas.
\end{itemize}

% 7 temos 66 skaidrė.
Išdėstymo diagramos aprašo fizinę sistemos architektūrą (susieja sistemos
komponentus su atitinkama technine įranga). Taip pat jos parodo, kaip
sistemos komponentai yra išdėstyti kompiuterių tinkluose.

% 7 temos 734 skaidrė.

% TODO Išsiaiškinti ir įdėti: 7 temos 735-780.

\subsubsection{Paketų diagramos}

\begin{itemize}
  \item Elementai.
  \item Notacija.
  \item Paskirtis.
  \item Prieigos keliai.
  \item Paketų tarpusavio priklausomybės.
  \item Paketų importavimas.
  \item Paketų suliejimas.
  \item Sistemos struktūros modeliavimas paketais.
  \item Modelių pakavimas.
\end{itemize}

% 7 temos 781 skaidrė.

% TODO Išsiaiškinti ir įdėti: 7 temos 782

\subsubsection{Papildomos priemonės}

\begin{itemize}
  \item Pastabos.
  \item Ribojimai.
  \item Komentarai.
  \item Žymėtosios reikšmės.
  \item Konstruktorių rūšys.
\end{itemize}
