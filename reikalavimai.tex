\section{Reikalavimai}

% 9 tema 3 skaidrė.

\Gls{sistemos_reikalavimas} – \glsentrydesc{sistemos_reikalavimas}.
Reikalavimai gali būti labai skirtingo abstrakcijos lygmens – nuo 
labai bendrų sistemos ar jos teikiamų paslaugų ribojimų iki detalios
matematinės sistemos savybių specifikacijos. To išvengti nėra įmanoma
dėl dvigubos reikalavimų paskirties:
\begin{itemize}
  \item viena vertus, jie naudojami skelbiant konkursą sistemai sukurti
    ir todėl turi būti pakankamai bendro pobūdžio, kad konkurse galėtų
    dalyvayti kuo daugiau pretendentų;
  \item kita vertus, jie yra pagrindinė sandorio tarp užsakovo ir vykdytojo
    dalis ir todėl turi būti suformuluoti kuo tiksliau ir išsamiau.
\end{itemize}

\Gls{reikalavimu_nustatymas} – \glsentrydesc{reikalavimu_nustatymas}.

Reikalavimų rūšys:
\begin{itemize}
  \item \emph{naudotojo reikalavimai (operaciniai poreikiai)} – turi 
    taip aprašyti reikalavimus, kad jie būtų suprantami žmonėms, tik
    paviršutiniškai susipažinusiems su tuo, kas yra kompiuteriai ir
    programų sistemos (šie reikalavimai skirti būsimiesiems sistemos
    naudotojams bei jos užsakovams);
  \item \emph{sistemos reikalavimai} – tam tikru specialiu būdu 
    struktūrizuotas detalus sistemos teikiamų paslaugų ir jos tenkinamų
    ribojimų aprašas (rašomi, kaip sudėtinė užsakovo ir vykdytojo
    sandorio dalis);
  \item \emph{projektiniai reikalavimai} – abstraktus programų sistemos
    įgyvendinimo būdo aprašas, naudjamas, kaip išeities ribojimai detaliai
    projektuojant sistemą (skirti vykdytojams – susieja sistemos 
    reikalavimus su jos realizacija).
\end{itemize}

\Gls{reikalavimu_formulavimas} – \glsentrydesc{reikalavimu_formulavimas}.

\Gls{reikalavimu_specifikacija} – \glsentrydesc{reikalavimu_specifikacija}.

Gerai suformuluoto reikalavimo savybės:
\begin{itemize}
  \item \emph{abstraktus} – specifikuoja operacinę (stebimą iš išorės)
    sistemos savybę ir nieko nekalba apie tai, kaip tą savybę realizuoti
    sistemoje (juodosios dėžės principas);
  \item \emph{išsamus} – turi prasmę ne tik tada, kai yra nagrinėjamas
    kartu su kitais reikalavimais, bet ir tuomet, kai jis nagrinėjamas
    atskirai;
  \item \emph{tikslus} – visi jame naudojami terminai turi griežtai
    apibrėžtas reikšmes;
  \item \emph{vienareikšmis} – jį galima interpretuoti vieninteliu būdu;
  \item \emph{verifikuojamas} – žinomas ir prieinamas baigtinis ir kainos
    bei kitais požiūriais priimtinas procesas, kurį taikant galima
    nustatyti, ar reikalavimas tikrai yra įgyvendintas;
  \item \emph{įgyvendinamas} – žinomas ir prieinamas toks ekonominiu,
    juridiniu bei kitais požiūriais priimtinas technologinis procesas,
    kurio inovaciniai slenksčiai gali būti pašalinti per priimtiną
    laikotarpį ir už priimtiną kainą ir kurį taikant galima sukurti
    sistemą, turinčią tuo reikalavimų specifikuojamą savybę;
  \item \emph{integruojamas} – sujungus jį su kitais reikalavimais yra
    gaunamas tarpusavyje suderintų reikalavimų rinkinys;
  \item \emph{lokalizuojamas} – jį galma susieti su vienu ar keliais
    konkrečiais kuriamos sistemos komponentais, įgyvendinančiais tą
    reikalavimą;
  \item \emph{trasuojamas} – vienareikšmiškai įvardinamas;
  \item \emph{unikalus} – jame neturi būti kartojama kituose reikalavimuose
    pateikta informacija;
  \item \emph{glaustas} – jame nėra pagrindimo, apibrėžčių ir kitų
    nebūtinų dalykų;
  \item \emph{suprantamas} – jame nėra vartojami tik specialistams 
    suprantami terminai ir yra aiškiai pasakyta, kokią funkcinę ar 
    nefunkcinę savybę privalo turėti sistema.
\end{itemize}

% 9 temos 21 skaidrė.

\Gls{funkcinis_reikalavimas} – \glsentrydesc{funkcinis_reikalavimas}.

\Gls{nefunkcinis_reikalavimas} – \glsentrydesc{nefunkcinis_reikalavimas}.
Nefunkciniai reikalavimai apima ribojimus sistemos paslaugų teikimo būdui
(pavyzdžiui, nurodoma maksimali užklausos trukmė), jos įgyvendinamoms
funkcijoms bei standartus, kurių privalu laikytis. Projekto valdymo
reikalavimai tiap pat yra nefunkciniai reikalavimai.

% 9 temos 56 skaidrė.

Veikimo ribojimai apima:
\begin{itemize}
  \item tikslumo;
  \item patikimumo – programų sistemos trykių neigiamo poveikio naudotojų
    verslo tikslams dydis;
  \item gyvybingumo – programų sistemos geba apsaugoti jos kritines
    funkcijas nuo trykių poveikio;
  \item robastiškumo – sugebėjimas atkurti savo prarastą funkcionalumą,
    ypač, praradus jį dėl klaidingų duomenų arba dėl kokių nors ypatingų
    situacijų;
  \item našumo
\end{itemize}
reikalavimus.

% 9 temos 86 skaidrė.
Quint – plačiai naudojamas programinės įrangos kokybės standarto
ISO 9126 plėtinys. Kokybės atributai:
\begin{itemize}
  \item \emph{funkcionalumas} (kokiu mastu sistemos funkcijos tenkina
    užsakovo (naudotojų) poreikius ir ar tos funkcijos turi pageidaujamas
    savybes):
    \begin{itemize}
      \item tinkamumas \eng{suitability} – sistemos funkcijų tinkamumas
        naudotojų poreikiams tenkinti ir jų išsamumas;
      \item tikslumas \eng{accuracy} – apibūdina, kokiu tikslumu sistema
        atlieka skaičiavimus bei kitus perdirbimus;
      \item kooperatyvumas \eng{interoperability} – sistemos gebėjimas
        kooperuotis su kitomis sistemomis;
      \item darnumas \eng{compliance} – apibūdina, kokiu mastu sistema
        yra suderinta su galiojančiais teisės aktais ir dalykinėje
        srityje galiojančiais standartais bei susitarimais;
      \item apsaugotumas \eng{security} – apibūdina sistemos gebėjimą
        apsisaugoti nuo tyčinio ar netyčinio neteisėto jos funkcionalumo
        ar joje saugomų duomenų panaudojimo;
      \item trasuojamumas \eng{traceability} – apibūdina pastangas,
        kurių reikia tam, kad pasirinktuose programos taškuose būtų
        galima patikrinti tarpinių skaičiavimų rezultatų teisingumą;
    \end{itemize}
  \item \emph{patikimumas} \eng{reliability} (programų sistemos gebėjimas,
    tinkamai eksploatuojant tą sistemą, specifikacijoje nurodytą laiko
    tarpą išlikti korektiška ir produktyvia):
    \begin{itemize}
      \item išbaigtumas \eng{maturity} – programoje esančių klaidų 
        sukeliamų trykių dažnis;
      \item atsparumas trykiams \eng{fault tolerance} – apibūdina, kokiu
        mastu sistema geba išlikti korektiška ir produktyvi, po
        trykių ar įsilaužimų;
      \item atkuriamumas \eng{recoverability} – apibūdina pastangas,
        reikalingas atkurti po trykio prarastą sistemos funkcionalumą
        ir/arba duomenis;
      \item prieinamumas \eng{availability} – laikas, per kurį galima
        pasinaudoti sistema, jei jos prireikia;
      \item pažeidžiamumas \eng{degradability} – pastangos, reikalingos
        atkurti esmines sistemos funkcijas, jai praradus savo 
        funkcionalumą;
    \end{itemize}
  \item \emph{perkeliamumas} \eng{portability} (galimybė kelti programų
    sistemą iš vienos platformos į kitą):
    \begin{itemize}
      \item adaptuojamumas \eng{adaptability} – apibūdina galimybę perkelti
        sistemą į kitą platformą, nedarant tos sistemos pakeitimų;
      \item įrašymo paprastumas \eng{installability} – apibūdina sistemai 
        įrašyti reikalingas pastangas;
        % FIXME Aiškiai parodyti skirtumą:
        % įrašymas – procesas, kurio metu sistemą įrašoma į kompiuterį;
        % įdiegimas – didesnis procesas, kuris apima įrašymą bei visokius
        % darbuotojų apmokymus ir panašiai.
      \item atitikimas standartams \eng{conformance} – apibūdina sistemos
        atitikimą perkeliamumo standartams ir susitarimams;
      \item pakeičiamumas \eng{replaceability} – apibūdina pastangas,
        reikalingas anksčiau naudotą kitą sistemą pakeisti naująja sistema;
    \end{itemize}
  \item \emph{panaudojamumas} (pastangos, kurių reikia naudotojams
    pasinaudoti sistema):
    \begin{itemize}
      \item suprantamumas \eng{understandability} – pastangos, kurių
        reikia naudotojui, kad perprastų sistemos koncepciją ir jos
        naudojimo būdą;
      \item išmokstamumas \eng{learnability} – pastangos, kurių 
        naudotojui reikia išmokti naudotis sistema;
      \item operabilumas \eng{operability} – kiek naudotojo pastangų
        reikia dirbant su sistema;
      \item būsenos vizualizavimas \eng{explicitness} – vertina sistemos
        gebėjimą informuoti apie tai, kas su ja vyksta;
      \item individualizuojamumas \eng{customisability} – galimybės
        prisitaikyti sistemą savo individualiems poreikiams;
      \item patrauklumas \eng{attractivity} – vertina, kokiu mastu
        sistemos teikiamos paslaugos, jos elgsena bei jos naudojami
        informacijos vizualizavimo bei pateikties būdai tenkina 
        išreikštiniu būdu nesuformuluotus naudojo pageidavimus bei
        nuostatas.
      \item aiškumas \eng{clarity} – vertina, kokiu mastu naudotojui
        akivaizdu, ką moka daryti sistema;
      \item informatyvumas \eng{helpfulness} – vertina, kokiu mastu
        naudotojams yra prieinama informacinė medžiaga;
      \item patogumas naudotojui \eng{user-friendliness} – vertina
        naudotojo pasitenkinimą sistema;
    \end{itemize}
  \item \emph{našumas} \eng{efficiency} (programų sistemos veikimo greičio
    ir, prie nurodytų sąlygų, jos naudojamų resursų santykis):
    \begin{itemize}
      \item našumas pagal laiką \eng{time behaviour} – apima sistemos
        reakcijos laiką, skaičiavimų trukmę ir jos pralaidumą
        (gebėjimą per tam tikrą laiką apdoroti tam tikrą įvykių skaičių);
      \item našumas pagal resursus \eng{resource behaviour} – kiek
        resursų ir kiek laiko naudoja sistema vykdydama savo funkcijas;
    \end{itemize}
  \item \emph{prižiūrimumas} \eng{maintainability} (pastangos, reikalingos
    sistemai perdaryti):
    \begin{itemize}
      \item analizuojamumas \eng{analysability} – pastangos, kurių
        reikia sistemos defektų ar trykių priežastims nustatyti bei
        išsiaiškinti, kurią sistemos dalį reikia keisti;
      \item keičiamumas \eng{changeability} – pastangos, kurių reikia 
        sistemos klaidai pašalinti ar padaryti pakeitimus, pritaikant
        sistemą jos aplinkos pokyčiams;
      \item stabilumas \eng{stability} – pakeitimų iššaukiamų, netikėtų
        efektų rizika;
      \item testuojamumas \eng{testability} – pastangos, kurių reikia
        sistemos korektiškumui (pradžioje ar padarius pakeitimus)
        patikrinti;
      \item valdomumas \eng{manageability} – pastangos, kurių reikia
        padaryti sistemą veikiančia ar jos veikimui atkurti;
      \item tiražuojamumas \eng{reusability} – vertina, kokiu mastu sistemą
        ar jos dalis galima pakartotinai panaudoti kituose projektuose.
    \end{itemize}
\end{itemize}

% TODO Suprasti ir įvesti: 9 temos 142-166.
