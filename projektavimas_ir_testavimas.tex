\section{Projektavimas ir testavimas}

% 11 tema 10 skaidrė.

Koncepcinis projektavimas – programų sistemos projektavimas aukščiausiame
abstrakcijos lygmenyje.

Eskizinis projektavimas – koncepcinio projektavimo rezultatai patikslinami,
atsižvelngiant į konkrečią kompiuterinę platformą.

Detalusis projektavimas – išreiškia eskizinį projektą realizavimo
infrastruktūros terminais.

Programų sistemos architektūros aprašas aprašo:
\begin{itemize}
  \item bendrą sistemos organizavimo būdą (konstrukcijos santykį);
  \item sistemos struktūrinius elementus ir jų sąsajas;
  \item struktūrinių ir elgsenos elementų komponavimą į posistemius;
  \item architektūros stilių, tai yra taisykles, kuriomis vadovaujantis
    bet kurio lygmens sistemos komponentai jungiami vienas su kitu.
\end{itemize}

% 11 temos 21 skaidrė.

Sistemos hierarchiją galima projektuoti panaudojant skirtingus 
dekompozicijos tipus:
\begin{itemize}
  \item funkcinę dekompoziciją;
  \item objektinę dekompoziciją;
  \item dekompoziciją į paslaugas \eng{service-oriented};
  \item dekompoziciją į užduotis \eng{task-oriented};
  \item …
\end{itemize}
