\chapter{Išorinė analizė}

\section{Poreikių analizė}

\label{tema:poreikiu_analize}

(6 tema, 12 skaidrė)

Užsakovo poreikių analizavimas:
\begin{enumerate}
  \item analitikas bendraudamas su užsakovu išsiaiškina, ko reikia jo
    veiklai – užsakovo operacinius poreikius (\gls{operacinis_poreikis})
    ir iš jų suformuluoja operacinius reikalavimus 
    (\gls{operacinis_reikalavimas});
  \item analitikas iš operacinių reikalavimų suformuluoja konkrečius
    reikalavimus programų sistemai (\gls{reikalavimas_programu_sistemai}).
\end{enumerate}

Užsakovo poreikių analizės metu yra rašoma verslo poreikių specifikacija.
Šios specifikacijos paskirtis yra aprašyti pačias bendriausias operacines
būsimos sistemos savybes – kaip sistema bus naudojama versle bei kokią
naudą iš to turės verslas. Ji dažniausiai yra rengiama prieš nusprendžiant 
ar apskritai yra prasminga vykdyti tokį projektą.
\begin{note}
  Kadangi verslo poreikių specifikacija yra koncepcinio lygmens dokumentas,
  tai jame neturi būti kalbama nei apie sistemos projektavimo, nei apie
  jos įgyvendinimo būdus.
\end{note}

Specifikacijoje aprašoma:
\begin{itemize}
  \item sistemos paskirtis (verslo terminais);
  \item verslo poreikiai (operaciniais reikalavimais);
  \item planuojamą sistemos naudojimo scenarijų;
  \item sistemos aplinką (TODO: Išsiaiškinti – 6 tema, 22 skaidrė);
  \item pirminę sistemos įgyvendinamumo analizę.
\end{itemize}

\subsection{Verslo poreikių specifikavimas}

Siekiant išsiaiškinti verslo poreikius pirmiausia yra nustatomi jo:
\begin{itemize}
  \item siekis – kokia organizacijos (užsakovo) paskirtis, ko ji siekia?
  \item vizija – kuo organizacija (užsakovo) norėtų tapti ateityje?
\end{itemize}

% FIXME Kaip visa tai siejasi hierarchiniu požiūriu? Vidinė ir išorinė
% analizės yra SSGG analizės dalys, ar tai, kas yra vykdoma po jos?

\subsubsection{SSGG analizė}

Siekiant kokybiškai suformuluoti viziją ir parengti kokybišką strategiją
gali būti naudojama SSGG \eng{SWOT} analizė. Jos metu yra 
nustatomi aplinkos veiksniai bei galimas jų poveikis verslui. Veiksniai, 
verslo požiūriu, gali būti:
\begin{itemize}
  \item vidiniai
    \begin{itemize}
      \item verslo stipriosios pusės \eng{Strengths} – turimi resursai
        ir galimybės, kuriomis galima pasinaudoti siekiant konkurencinių
        pranašumų (pavyzdžiui, patentai, gera reputacija tarp klientų);
      \item verslo silpnosios pusės \eng{Weaknesses} – stipriųjų pusių
        nebuvimas (pavyzdžiui, bloga reputacija tarp klientų, aukšta
        gamybos savikaina);
    \end{itemize}
  \item išoriniai
    \begin{itemize}
      \item galimybės \eng{Opportunities} – bet kurie išoriniai verslo
        proceso veiksniai, potencialiai padedantys tam procesui
        greičiau ar efektyviau siekti norimų verslo tikslų;
      \item grėsmės \eng{Threats} – bet kurie išoriniai verslo proceso
        veiksniai, kaip nors ribojantys to proceso galimybes ar 
        sukuriantys kliūtis pasiekti norimus verslo tikslus.
    \end{itemize}
\end{itemize}

% FIXME Išsiaiškinti (6 tema, 51 skaidrė).
%Tikslai gali būti formuluojami keturiuose sektoriuose:
%\begin{itemize}
  %\item \emph{įeigos logistika} – tiekėjai, tiekimo problemos, 
    %tiekimo grėsmės;
  %\item \emph{išeigos logistika} – platintojai, perpardavinėtojai,
    %užsakovai, klientai, platinimo bei pardavimo problemos ir grėsmės;
  %\item \emph{teisinis reguliavimas} – teisiniai ir privatumo
    %apsaugos klausimai, su jais siejamos problemos ir grėsmės;
  %\item \emph{išorinis vertinimas (įvaizdis)} – vertintojai, įvaizdžio
    %problemos ir grėsmės.
%\end{itemize}

\subsection{Išorinė analizė}

Atlikus SSGG analizę turėtų būti atliekama išorinė verslo proceso
analizė, kuria siekiama sukurti kriterijų (verslo sėkmės matų sistemą), 
kuriuo galima vertinti verslo efektyvumą. Paskui kiekvienam matui
yra priskiriama po kritinę reikšmę – jei reali reikšmė yra artima kritinei,
tai laikoma grėsme, o jei viršija – problema.
\begin{exmp}
  Sėkmės matų pavyzdžiai:

  \begin{tabularx}{\textwidth}{l|X|X}
    Veiksnys & Apibrėžimas & Formulė \\
    \hline
    Pelningumas & Įplaukų ir kaštų santykis & 
      $\frac{\text{pajamos}}{\text{išlaidos}}$ \\
    Produktyvumas & Išeigos ir sunaudotų resursų santykis &
      $\frac{\text{produktai}}{%
      \text{darbo sanaudos} + \text{sunaudotos žaliavos}}$ \\
    Efektyvumas & Tikslų įgyvendinimo laipsnis &
      $\frac{\text{faktinė gamybos apimtis per laiko vienetą}}{%
      \text{planuota gamybos apimtis per laiko vienetą}}$
  \end{tabularx}

  Matavimų pavyzdžiai:

  \begin{tabularx}{\textwidth}{l|X|X|X}
    Veiksnys & Analizės rezultatai & Vertinimo kriterijus & Diagnozė \\
    \hline
    Produktyvumas & $85 \%$ & $\geq 80 \%$ & Grėsmė \\
    Efektyvumas & $75 \%$ & $\geq 80 \%$ & Problema \\
  \end{tabularx}
\end{exmp}

\subsection{Vidinė analizė}

Baigus išorinę analizę yra atliekama visi verslo sistemos analizė. Jos
metu siekiama išsiaiškinti verslo stipriąsias ir silpnąsias puses,
vidines išorinės analizės metu nustatytų problemų bei grėsmių priežastis.
Taip pat kokiomis neišnaudotomis galimybėmis verslas yra pajėgus 
pasinaudoti. Populiariausi vidinės analizės metodai:
\begin{itemize}
  \item išteklių auditas;
  \item kaštų ir pelno analizė;
  \item lyginamoji analizė \eng{benchmarking};
  \item vertės grandinės analizė;
  \item tiekimo grandinės analizė.
\end{itemize}

\subsection{Strategija}

\Gls{strategija} – \glsentrydesc{strategija}. Strategijos formavimo
žingsniai:
\begin{enumerate}
  \item esamos padėties analizė (grėsmės, problemos, galimybės);
  \item vizijos formulavimas (kokias problemas bei grėsmes šalinti);
  \item strategijos formulavimas (kaip įgyvendinti viziją, kokiomis
    naujomis galimybėmis pasinaudoti);
  \item strategijos realizavimas (tikslų medis).
\end{enumerate}

Sustota: 93 skaidrė.
