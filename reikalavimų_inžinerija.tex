\chapter{Reikalavimų inžinerija}

Ką sistema turi daryti (funkcijos) ir kaip ji tai turi daryti (ribojimai)
yra aprašomi sistemos reikalavimais. \Gls{sistemos_reikalavimas} tai yra 
\glsentrydesc{sistemos_reikalavimas}, o \gls{reikalavimu_inzinerija} –
tai \glsentrydesc{reikalavimu_inzinerija}.

Pagrindinis reikalavimų inžinerijos uždavinys yra transformuoti operacinius
poreikius (naudotojų reikalavimus) į reikalavimus programų sistemai.
Reikalavimų inžinerija apima:
\begin{itemize}
  \item poreikių analizę;
  \item reikalavimų analizę;
  \item reikalavimų specifikavimą.
\end{itemize}

Būtina suvokti, kad iš tiesų užsakovui reikia ne kompiuterių, kompiuterinių
tinklų ir netgi ne programų sistemų. Jam reikia tam tikro efekto, kurį
sukuria techninės ir programinės įrangos kompleksas. Jei kalbame apie
verslo įmones, tai joms rūpi patobulinti savo verslą. Todėl, 
kompiuterizuojant įmonę, darbą reikia pradėti ne nuo konkrečių programų
sistemos reikalavimų, o pirma reikia išsiaiškinti su kokiomis problemomis
susiduria verslas, kaip jas galima būtų išspręsti naudojantis programų 
sistemomis.

TODO: Išsiaiškinti „Priklausomybių modelis“ 6 tema, 10 skaidrė.
