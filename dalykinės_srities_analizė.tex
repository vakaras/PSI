\section{Dalykinės srities analizė}

% 10 temos 2 skaidrė.

Dalykinės srities (arba verslo) analizė yra pirmoji informacinių sistemų
kūrimo proceso stadija. Tai taip pat pirmoji verslo sistemų reinžinerijos
ir kokybės valdymo procesų stadija. Dalykinės srities analizė turi
būti atliekama prieš pradedant bet kuriuos organizacijos pertvarkymus,
nepaisant to, kokiame lygmenyje ir dėl kokių priežasčių jie yra daromi.

Kuriant sistemą „iš viršaus žemyn“, pirmose stadijose turi būti 
samprotaujama viršutinių abstrakcijos lygmenų sąvokomis. Kitaip tariant,
analitikas visų pirma turi susidaryti bendrą, apžvalginį, visos įmonės
vaizdą ir tik po to pradėti gilintis į konkrečių darbuotojų veiklas
bei užduotis. Konkrečios veiklos ir užduotys visuomet turi būti
analizuojamos viso verslo kontekste. Viso verslo kontekstas turi būti
naudojamas taip pat ir vertinant visus atliekamos analizės rezultatus.

Sisteminis analitikas – žmogus, kuris nustato projekto apimtis, tikslus,
o daugeliu atveju ir jo terminus. Sisteminio analitiko vaidmenys:
\begin{itemize}
  \item \emph{reporteris} – asmuo (paprastai, žurnalistas), kurio 
    pagrindinė užduotis yra kuo objektyviau aprašyti su kokiu nors įvykiu
    susijusius faktus ir detales;
  \item \emph{detektyvas} – asmuo, kurio pagrindinė užduotis yra 
    išsiaiškinti su kokiu nors įvykiu susijusius faktus ir nustatyti
    atsakingus asmenis;
  \item \emph{konsultantas} – asmuo, kurio pagrindinė užduotis yra padėti
    klientui, teikiant jam paslaugas, patarimus ar nurodymus, kaip
    jam vykdyti kokią nors užduotį;
  \item \emph{diagnostikas} – asmuo, kurio pagrindinė užduotis yra tirti
    faktus apie kokį nors įvykį ar apie kokią nors problemą ir,
    remiantis tyrimo rezultatais ir savo patirtimi, nustatyti to
    įvykio ar problemos priežastis;
  \item \emph{tyrėjas} – asmuo, kurio darbo pobūdis yra panašus į 
    detektyvo, tik jo tyrimų sritis paprastai yra siauresnė ir jis
    dažniausiai nesidomi atsakomybės klausimais;
  \item \emph{organizatorius} – asmuo, kurio pagrindinė užduotis yra 
    planuoti veiklų ir įvykių sekas, prisiimant pilną atsakomybę už
    teisingą tų veiklų vykdymą ir vykstančius įvykius;
  \item \emph{galvosūkių sprendėjas} – asmuo, kuris iš atskirų gabaliukų
    geba sudėlioti visumą arba sugeba rasti sprendimą, vadovaudamasis
    tik užuominomis ir pavienėmis detalėmis;
  \item \emph{vertintojas} – asmuo, kuris tikrina, reitinguoja bei 
    vertina faktus, jų tikslumo, išsamumo ir korektiškumo požiūriais ir
    duoda jiems kompleksinius įverčius;
  \item \emph{paprastintojas} – asmuo, kuris nagrinėja sudėtingus
    objektus, įvykius ar procesus ir skaido juos į paprastesnes dalis.
  \item \emph{genties žvalgas} – asmuo, kuris pirmasis atsiranda ten,
    kur juda gentis, žiūri, ar jos nelaukia kokie nors netikėti pavojai
    ir ieško geriausio kelio per džiungles;
  \item \emph{menininkas} – asmuo, interpretuojantis faktus ir įvykius
    bei randantis jų paslėptą prasmę (moka pavaizduoti daiktus kokie
    jie yra, kokie jie turėtų būti ar kokiais juos kas nors norėtų
    matyti; meninkas gali dirbti ne tik su esama tikrove, bet ir formuoti
    norimos tikrovės vaizdą);
  \item \emph{skulptorius} – menininkas, suteikiantis prasmingas formas
    mažai vertingos ar netgi bevertės medžiagos gabalams.
\end{itemize}

Sisteminio analitiko užduotys:
\begin{itemize}
  \item pradinėse projekto stadijose nurodyti veiklos kryptis vykdytojų
    kolektyvui;
  \item detalizuoti „verslo žinias“;
  \item perprasti naudotojų problemas ir pažvelgti į jas naudotojų akimis;
  \item išryškinti ir suformuluoti verslo problemas;
  \item pasiūlyti, kaip praktiškai ir kuo pigiau spręsti užsakovui svarbias
    verslo problemas;
  \item pasiūlyti alternatyvas, kaip verslo problemų sprendimą palaikyti
    IT priemonėmis;
  \item prieš pradedant sistemos realizavimą, išnagrinėti sistemos
    funkcijų įgyvendinamumo ir sistemos įdiegiamumo klausimus ir pasiūlyti,
    kaip spręsti galimas problemas;
  \item išsiaiškint, ką dar reikia papildomai panagrinėti, atlikti tokius
    nagrinėjimus ir dokumentuoti jų rezultatus;
  \item sekti, kad vykdytojų kolektyvas nenukryptų nuo projekto tikslų;
  \item tiksliai ir suprantamai dokumentuoti operacinius naudotojų
    poreikius;
  \item perteikti programų sistemų inžinieriams informaciją apie naudotojų
    poreikius ir specifikuoti kuriamos sistemos reikalavimus;
  \item pristatyti projektą užsakovui ir savo vadovybei;
  \item sekti, kad vykdant projektą būtų laikomasi organizacijos
    standartų ir procedūrų;
  \item sketi, kad projekto rezultatai tenkintų sandorio ir kitus
    privalomus reikalavimus.
\end{itemize}

% 10 temos 26 skaidrė.
Verslo sistemos funkcija (funkcinė sritis) – tai tam tikra verslo sistemos
veiklos sritis. Verslo funkcija turi būti:
\begin{itemize}
  \item įvardinama ir nusakoma, bet nebūtinai matuojama;
  \item ji turi būti apibrėžiama veiklų, atsakomybių ir atskaitingumo
    terminais.
\end{itemize}

Bet kurios organizacijos funkcijos gali būti suskirstytos į dvi
kategorijas:
\begin{itemize}
  \item \emph{tiesiogines verslo funkcijas} – veiklas, tiesiogiai
    susijusias su gamyba, paslaugų teikimu, pelno gavimu arba tokių
    veiklų valdymu;
  \item \emph{administravimo, valdymo, apskaitos ir kontrolės funkcijas}
    – funkcijas, skirtas organizacijos, kaip juridinio asmens, kasdieniam
    darbui organizuoti.
\end{itemize}

% TODO Suprasti ir įdėti: 10 temos 35-54.

% 10 temos 55 skaidrė.

Verslo procesas – tai tarpusavyje susijusių veiklų seka. 

Veikla – tai įvardinama verslo proceso dalis, kurianti kokį nors
prasmingą tarpinį verslo proceso rezultatą. Veiklą sudaro tiksliai
nusakomomis procedūromis išreiškiamų užduočių seka.

\begin{note}
  Funkcijos ir subfunkcijos yra valdomos, o veiklos yra vykdomos.
\end{note}

% 10 temos 65 skaidrė.

Analizuojant verslo procesus ir veiklas, atliekama gilesnė funkcinės
analizės metu analizuotų verslo funkcijų ir su jomis susietos
informacijos analizė.

Verslo funkcijos yra tai, dėl ko ir egzistuoja įmonė. Verslo funkcijas
sudaro procesų ir galbūt paskirų veiklų rinkiniai – vykdant veiklų ir
individualių veiklų derinius ir yra atliekamas verslo funkcija
numatytas darbas. Tuo tarpu verslo procesai yra sudaryti iš veiklų, o
veiklos – iš užduočių.

% TODO Suprasti ir įvesti 66-88 skaidrė.

% 10 temos 88 skaidrė.

Pagrindiniai informacijos rinkimo metodai:
\begin{itemize}
  \item interviu – analitiko pokalbis su užsakovu, dalykinės srities
    specialistu, kuriamos sistemos naudotoju ar kitu asmeniu, kurio
    metu yra atsakoma į iš anksto parengto klausimyno klausimus ir
    visi atsakymai yra išsamiai dokumentuojami:
    \begin{itemize}
      \item tiesioginis – 
      \item telefoninis – naudojamas renkant atsakymus į klausimus, į
        kuriuos galima atsakyti „taip“ arba „ne“;
    \end{itemize}
  \item apklausa – naudojama, kuomet reikia surinkti iš didelio skaičiaus
    asmenų daug, kiekybinių duomenų;
  \item fokuso grupės (pasitarimas) – naudojamos gauti skirtingas reakcijas 
    į kokį nors vieną klausimą;
  \item darbo vietų lankymas – naudojamas, norint tiesiogiai susipažinti
    su darbo procesais, užduočių vykdymu, darbo aplinka ir darbo sąlygomis
    (vienas iš tiesioginio stebėjimo būdų);
  \item bendros darbo grupės – dalykinės srities specialistai yra suburiami
    į vieną grupę, vadovaujamą analitiko-moderatoriaus, ir patys analizuoja
    verslo funkcijas, procesus, veiklas bei duomenis;
    % 10 temos 151 skaidrė.
  \item maketų panaudojimas (sistemai kurti arba vertinti, ypač sistemos
    patogumo naudotojams vertinimui);
  \item rašytinių šaltinių analizė;
  \item stebėjimas.
\end{itemize}

