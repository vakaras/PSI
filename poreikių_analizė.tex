% \chapter{Išorinė analizė} 
% FIXME Kodėl „Poreikių analizė“ nurodyta, kaip „Išorinės analizės“
% poskyris?

\section{Poreikių analizė}

\label{tema:poreikiu_analize}

(6 tema, 12 skaidrė)

Užsakovo poreikių analizavimas:
\begin{enumerate}
  \item analitikas bendraudamas su užsakovu išsiaiškina, ko reikia jo
    veiklai – užsakovo operacinius poreikius (\gls{operacinis_poreikis})
    ir iš jų suformuluoja operacinius reikalavimus 
    (\gls{operacinis_reikalavimas});
  \item analitikas iš operacinių reikalavimų suformuluoja konkrečius
    reikalavimus programų sistemai (\gls{reikalavimas_programu_sistemai}).
\end{enumerate}

Užsakovo poreikių analizės metu yra rašoma verslo poreikių specifikacija.
Šios specifikacijos paskirtis yra aprašyti pačias bendriausias operacines
būsimos sistemos savybes – kaip sistema bus naudojama versle bei kokią
naudą iš to turės verslas. Ji dažniausiai yra rengiama prieš nusprendžiant 
ar apskritai yra prasminga vykdyti tokį projektą.
\begin{note}
  Kadangi verslo poreikių specifikacija yra koncepcinio lygmens dokumentas,
  tai jame neturi būti kalbama nei apie sistemos projektavimo, nei apie
  jos įgyvendinimo būdus.
\end{note}

Specifikacijoje aprašoma:
\begin{itemize}
  \item sistemos paskirtis (verslo terminais);
  \item verslo poreikiai (operaciniais reikalavimais);
  \item planuojamą sistemos naudojimo scenarijų;
  \item sistemos aplinką (TODO: Išsiaiškinti – 6 tema, 22 skaidrė);
  \item pirminę sistemos įgyvendinamumo analizę.
\end{itemize}

\subsection{Verslo poreikių specifikavimas}

Poreikių analizės planas:
\begin{enumerate}
  \item išorinė analizė – nustatomos problemos, grėsmės ir galimybės;
  \item vidinė analizė – nustatomos problemų priežastys;
  \item vizija – sugalvojama, kokiu turėtų tapti verslas;
  \item tikslų medis – sugalvojama, kaip vizija turėtų būti įgyvendinta;
  \item operaciniai poreikiai – nustatoma, kokių IT paslaugų reikia;
  \item scenarijus – sugalvojama, kaip turėtų būti naudojamasi sistema.
  \item įgyvendinimo plano sudarymas – sugalvojama, kaip turėtų įgyvendinti
    numatyti pakeitimai.
\end{enumerate}

Siekiant išsiaiškinti verslo poreikius pirmiausia yra nustatomi jo:
\begin{itemize}
  \item siekis – kokia organizacijos (užsakovo) paskirtis, ko ji siekia?
  \item vizija – kuo organizacija (užsakovo) norėtų tapti ateityje?
\end{itemize}

% FIXME Kaip visa tai siejasi hierarchiniu požiūriu? Vidinė ir išorinė
% analizės yra SSGG analizės dalys, ar tai, kas yra vykdoma po jos?

\subsubsection{SSGG analizė}

Siekiant kokybiškai suformuluoti viziją ir parengti kokybišką strategiją
gali būti naudojama SSGG \eng{SWOT} analizė. Jos metu yra 
nustatomi aplinkos veiksniai bei galimas jų poveikis verslui. Veiksniai, 
verslo požiūriu, gali būti:
\begin{itemize}
  \item vidiniai
    \begin{itemize}
      \item verslo stipriosios pusės \eng{Strengths} – turimi resursai
        ir galimybės, kuriomis galima pasinaudoti siekiant konkurencinių
        pranašumų (pavyzdžiui, patentai, gera reputacija tarp klientų);
      \item verslo silpnosios pusės \eng{Weaknesses} – stipriųjų pusių
        nebuvimas (pavyzdžiui, bloga reputacija tarp klientų, aukšta
        gamybos savikaina);
    \end{itemize}
  \item išoriniai
    \begin{itemize}
      \item galimybės \eng{Opportunities} – bet kurie išoriniai verslo
        proceso veiksniai, potencialiai padedantys tam procesui
        greičiau ar efektyviau siekti norimų verslo tikslų;
      \item grėsmės \eng{Threats} – bet kurie išoriniai verslo proceso
        veiksniai, kaip nors ribojantys to proceso galimybes ar 
        sukuriantys kliūtis pasiekti norimus verslo tikslus.
    \end{itemize}
\end{itemize}

% FIXME Išsiaiškinti (6 tema, 51 skaidrė).
%Tikslai gali būti formuluojami keturiuose sektoriuose:
%\begin{itemize}
  %\item \emph{įeigos logistika} – tiekėjai, tiekimo problemos, 
    %tiekimo grėsmės;
  %\item \emph{išeigos logistika} – platintojai, perpardavinėtojai,
    %užsakovai, klientai, platinimo bei pardavimo problemos ir grėsmės;
  %\item \emph{teisinis reguliavimas} – teisiniai ir privatumo
    %apsaugos klausimai, su jais siejamos problemos ir grėsmės;
  %\item \emph{išorinis vertinimas (įvaizdis)} – vertintojai, įvaizdžio
    %problemos ir grėsmės.
%\end{itemize}

\subsection{Išorinė analizė}

Atlikus SSGG analizę turėtų būti atliekama išorinė verslo proceso
analizė, kuria siekiama sukurti kriterijų (verslo sėkmės matų sistemą), 
kuriuo galima vertinti verslo efektyvumą. Paskui kiekvienam matui
yra priskiriama po kritinę reikšmę – jei reali reikšmė yra artima kritinei,
tai laikoma grėsme, o jei viršija – problema.
\begin{exmp}
  Sėkmės matų pavyzdžiai:

  \begin{tabularx}{\textwidth}{l|X|X}
    Veiksnys & Apibrėžimas & Formulė \\
    \hline
    Pelningumas & Įplaukų ir kaštų santykis & 
      $\frac{\text{pajamos}}{\text{išlaidos}}$ \\
    Produktyvumas & Išeigos ir sunaudotų resursų santykis &
      $\frac{\text{produktai}}{%
      \text{darbo sanaudos} + \text{sunaudotos žaliavos}}$ \\
    Efektyvumas & Tikslų įgyvendinimo laipsnis &
      $\frac{\text{faktinė gamybos apimtis per laiko vienetą}}{%
      \text{planuota gamybos apimtis per laiko vienetą}}$
  \end{tabularx}

  Matavimų pavyzdžiai:

  \begin{tabularx}{\textwidth}{l|X|X|X}
    Veiksnys & Analizės rezultatai & Vertinimo kriterijus & Diagnozė \\
    \hline
    Produktyvumas & $85 \%$ & $\geq 80 \%$ & Grėsmė \\
    Efektyvumas & $75 \%$ & $\geq 80 \%$ & Problema \\
  \end{tabularx}
\end{exmp}

\subsection{Vidinė analizė}

Baigus išorinę analizę yra atliekama visi verslo sistemos analizė. Jos
metu siekiama išsiaiškinti verslo stipriąsias ir silpnąsias puses,
vidines išorinės analizės metu nustatytų problemų bei grėsmių priežastis.
Taip pat kokiomis neišnaudotomis galimybėmis verslas yra pajėgus 
pasinaudoti. Populiariausi vidinės analizės metodai:
\begin{itemize}
  \item išteklių auditas;
  \item kaštų ir pelno analizė;
  \item lyginamoji analizė \eng{benchmarking};
  \item vertės grandinės analizė;
  \item tiekimo grandinės analizė.
\end{itemize}

\subsection{Strategija}

6 tema 93 skaidrė.

\Gls{strategija} – \glsentrydesc{strategija}. Strategijos formavimo
žingsniai:
\begin{enumerate}
  \item esamos padėties analizė (grėsmės, problemos, galimybės);
  \item vizijos formulavimas (kokias problemas bei grėsmes šalinti);
  \item strategijos formulavimas (kaip įgyvendinti viziją, kokiomis
    naujomis galimybėmis pasinaudoti);
  \item strategijos realizavimas (tikslų medis).
\end{enumerate}

Strategijos gali būti įvairių lygmenų:
\begin{itemize}
  \item \emph{korporacijos lygmens strategijos} – nusako kaip siekiama
    pertvarkyti visą verslą (ilgalaikės ir labai bendros strategijos);
  \item \emph{konkuravimo strategijos} – nusako kokiu būdu bus varžomasi
    su konkurentais;
  \item \emph{funkcinio lygmens strategijos} – nusako ko turi siekti
    organizacijos funkciniai padaliniai (buhalterija, atsiskaitymų
    skyrius).
\end{itemize}

Galima išskirti tris pagrindinius korporacijos lygmens strategijų tipus:
\begin{itemize}
  \item \emph{plėtros strategijos}:
    \begin{itemize}
      \item \emph{sutelkties} – verslas sutelkiamas pagrindinei (ką 
        geriausiai moka) gaminių (paslaugų) šeimai ir ieško augimo būdų
        didinant šios gamybos apimtis;
      \item \emph{vertikalaus integravimo} – verslas bando perimti
        įeigos logistikos (integravimas atgal) ar išeigos logistikos
        (integravimas pirmyn) kontrolę;
      \item \emph{horizontalaus integravimo} – verslas auga perimdamas
        varžovų įmones;
      \item \emph{plėtros} – verslas auga eidamas į naujas veiklos sritis;
      \item \emph{internacionalizavimo} – verslas yra 
        internacionalizuojamas panaudojant bet kurią iš integravimo ar
        plėtros strategijų, jei jo įsigyjamos bendrovės yra užsienyje;
    \end{itemize}
  \item \emph{stabilumo strategijos} – siekiama išlaikyti esamą dydį ir
    esamas operacijų apimtis:
    \begin{itemize}
      \item \emph{stabilus augimas} – verslo strategijoje nenumatoma jokių
        šuolių;
      \item \emph{atsitraukimas} \eng{pofit} 
        % FIXME Ar tikrai „pofit“? 6 temos 116 skaidrė.
        – atsisakoma ilgalaikių 
        tendencijų arba ilgą laiką gamintas produktas keičiamas nauju 
        (kadangi nereikalaujama papildomų investicijų, tai ši strategija 
        iš tiesų nėra augimo strategija). Galimos rūšys:
        \begin{itemize}
          \item \emph{pasipelnymo} \eng{harvest} – parduodamas senasis
            verslas ir imamasi kito verslo;
          \item \emph{žaidimo pabaigos} \eng{endgame} – iš dalies 
            liekama senajame versle „iki žaidimo pabaigos“, tuo pačiu
            metu po truputi pereinant į naują verslą;
        \end{itemize}
    \end{itemize}
  \item \emph{likvidavimo/atnaujinimo strategijos} – mažinami kaštai ir
    verslas restruktūrizuojamas:
    \begin{itemize}
      \item \emph{išlaidų mažinimo} \eng{retrenchment} – trumpalaikė 
        atnaujinimo strategija, kuria siekiama pašalinti silpnąsias verslo
        puses, iššaukusias verslo regresą;
      \item \emph{atgaivinimo} \eng{turnaround} – 
        % TODO Išsiaiškint 6 tema 118 skaidrė.
    \end{itemize}
    Verslo restrukūrizavimo būdai:
    \begin{itemize}
      \item \emph{praradimas} \eng{divestment} – tam tikra verslo dalis
        parduodama kitai (augančiai) bendrovei, kuri tęsia nusipirktą
        verslą;
      \item \emph{suskaldymas} \eng{spin-off} – dalis verslo tampa
        savarankiška bendrove;
      \item \emph{reinžinerija} – atsisakoma buvusių prielaidų apie tai,
        kaip turi veikti verslo procesai ir tie procesai visiškai
        perprojektuojami (dažnai netgi nuo nulio);
      \item \emph{mažėjimas} \eng{downsizing, layoff} – mažinamas 
        personalas nemažinant darbų apimties;
      \item \emph{likvidavimas};
      \item \emph{bankrotas}.
    \end{itemize}
\end{itemize}

Michaelis Porteris \org{Michael Porter} išskyrė trys galimas konkuravimo
strategijas:
\begin{itemize}
  \item \emph{kaštų lyderio strategija} – siekiama turėti kuo mažesnius
    gamybos bei pardavimo kaštus ir pardavinėti gaminius bei paslaugas
    pačiomis mažiausiomis kainomis.
  \item \emph{išskirtinumo strategija} – siekiama teikti unikalius,
    naudotojų ypač vertinamus ypatumus turinčius gaminius bei paslaugas;
  \item \emph{sutelkties strategija} – kaštų lyderio arba išskirtinumo
    strategija taikoma kokiam nors siauram rinkos segmentui (specialaus
    pobūdžio klientams).
\end{itemize}
Strategiją parenka analitikas, vadovaudamasis SSGG analizės rezultatais.

Informacines technologijas ir programų sistemas tikslinga naudoti
įgyvendinant šias strategijas:
\begin{itemize}
  % TODO Susieti su aprašymais.
  \item funkcinio lygmens strategijas;
  \item varžymosi strategijas;
  \item verslo augimo strategijas;
  \item stabilumo strategijas;
  \item verslo reinžinerijos strategijas.
\end{itemize}

\subsection{Tikslų medis}

% TODO Pertvarkyti.
Verslo vizija atsako į klausimą, ko mes siekiame pertvarkydami verslą.
Verslo tobulinimo strategija pasako kokiomis strateginėmis nuostatomis
vadovausimės pertvarkydami verslą. Dabar reikia nuspręsti, kokiais
būdais bus bandoma įgyvendinti tas strategines nuostatas. Kitaip
tariant, vizija turi būti detalizuota ir sukonkretinta, sukonstruojant
jai įgyvendinti skirtą tikslų medį.

\Gls{tikslu_medis} – \glsentrydesc{tikslu_medis}. Jis yra konstruojamas
atliekant tikslų dekompozicija: strateginiai tikslai yra išreiškiami
per taktinio lygmens tikslus, o pastarieji – per pamatuojamus operacinio
lygmens tikslus.

\begin{note}
  % TODO Išsiaiškinti iki galo. 6 temos 141 skaidrė.
  Verslo tikslai, operaciniai naudotojų poreikiai ir tiems poreikiams
  tenkinti reikalinga techninė bei programinė įranga yra du skirtingi
  dalykai ir turėtų būti griežtai vieni nuo kitų atskirti: verslo
  kompiuterizavimas nėra tikslas ir todėl jokie su tuo susiję dalykai
  (pavyzdžiui, sukurti interneto svetainę) neturėtų būti įtraukti į tikslų 
  medį.
\end{note}

% TODO Išsiaiškinti. 6 temos 145 skaidrė. 
%
% Operaciniai tikslai – tai žemiausio lygio verslo tikslai (jokio
% žodžio apie IT!)
%
% Operaciniai poreikiai – tai skaičiavimo arba informacinės paslaugos
% (dalykinės programos, duomenų bazės, interneto tinklalapiai ir kt.),
% padedančios naudotojams pasiekti numatytus operacinius tikslus.
%
% Kaip suprasti: „TĮ (pvz., kompiuterių tinklas) arba sisteminė PĮ
% (pvz., DBVS) nėra operaciniai poreikiai. Tai priemonės operaciniams
% tikslams įgyvendinti.“
% 
% Gal jis turėjo omeny, kad duomenų bazė yra tiesiog sutvarkyti duomenys
% – tai būtų operacinis poreikis, o konkreti DBVS jau būdas jai 
% įgyvendinti?
Turint operacinio lygmens tikslus galima su jais susieti operacinius
poreikius (\gls{operacinis_poreikis}) ir nustatyti priemones tiems
poreikiams įgyvendinti?

\subsection{Sistemos naudojimo scenarijus}

% 6 temos 147 skaidrė.

Sistemos naudojimo scenarijus aprašo kuriamą sistemą ir jos aplinką, 
įskaitant ja besinaudojančių agentų elgseną bei visą kontekstinę 
informaciją reikalingą pirminei įgyvendinamumo analizei atlikti.
Sistemos naudojimo scenarijuje aprašoma:
\begin{itemize}
  \item kaip bus dirbama organizacijoje po to, kai programų sistema bus
    sukurta ir įdiegta;
  \item patys bendriausi naudotojų sąsajų reikalavimai;
  \item patys bendriausi naudotojų darbo vietų reikalavimai.
\end{itemize}
Jame taip pat gali būti aprašoma:
\begin{itemize}
  \item kas daroma, atliekant rankines operacijas;
  \item sistemos sąveika su liktinėmis \eng{legacy} programų sistemomis;
  \item agentai, jų vaidmenys bei jų darbo aplinka.
\end{itemize}

\subsection{Įgyvendinimo planas}

% 6 tema 157 skaidrė.

Pakeitimų įgyvendinimo plane turi būti numatyti veiksmai, susiję su:
\begin{itemize}
  \item reikiamos techninės ir programinės įrangos pirkimu;
  \item kompiuterių tinklo kūrimu;
  \item personalo mokymu;
  \item instrukcijų, įsakymų ir kitų reikalingų dokumentų parengimu.
\end{itemize}

\subsection{Įgyvendinamumo analizė}

Įgyvendinamumo analizės tikslas yra įsitikinti, kad sistemą galima sukurti
ir kad kurti ją yra tikrai verta. Analizė turi atsakyti vadovybei
į šiuos klausimus:
\begin{itemize}
  \item ar projektas įgyvendinamas?
  \item kokiais alternatyviais būdais jį galima įgyvendinti?
  \item kokiais kriterijais vadovautis parenkant alternatyvą?
  \item kur alternatyva yra geriausia?
\end{itemize}
Analizės turinys:
\begin{itemize}
  \item ar mes žinome kaip kurti sistemą?
  \item ar mūsų žinių ir mokėjimų tam pakanka?
  \item ar turimų pinigų tam pakanka?
  \item ar galima sukurti per priimtiną laiką?
  \item ar tikrai bus iš to konkreti nauda?
\end{itemize}
Įgyvendinamumo aspektai:
\begin{itemize}
  \item operacinis:
    \begin{itemize}
      \item ar užsakovas pajėgus eksploatuoti sistemą?
      \item ar scenarijus iš tiesų veiks?
      \item ar naudotojai suinteresuoti laikytis scenarijumi nustatytų
        darbo taisyklių?
    \end{itemize}
  \item techninis:
    \begin{itemize}
      \item ar žinoma problemos sprendimo teorija ir ar yra ją palaikanti
        technologija?
      \item ar vykdytojai yra pajėgūs sukurti sistemą?
    \end{itemize}
  \item ekonominis:
    \begin{itemize}
      \item ar atsiperkamumo analizė pateisina projektą?
      \item ar grįš investicijos?
    \end{itemize}
  \item plano:
    \begin{itemize}
      \item ar galima su turimais vykdytojais ir kitais resursais
        pabaigti projektą laiku?
    \end{itemize}
  \item teisinis/etinis:
    \begin{itemize}
      \item ar projektas nepažeis kokių nors teisės ar pripažintų etinių
        normų?
    \end{itemize}
\end{itemize}
