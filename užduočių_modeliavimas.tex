\subsection{Užduočių modeliavimas}

% 8 temos 2 skaidrė.

Užduotis \eng{use case} apibrėžia scenarijų (\gls{scenarijus}), kuris 
aprašo, kaip \gls{agentas} naudojasi sistema savo specifiniams tikslams 
pasiekti. Užduočių modelį sudaro visų su sistema dirbančių agentų
ir visų tų agentų vykdomų užduočių visuma.

Užduočių modeliai:
\begin{itemize}
  \item padeda suprasti, kaip naudotojai įsivaizduoja būsimos sistemos
    funkcinius reikalavimus;
  \item yra išeities taškas, nuo kurio pradedama nustatinėti, kokios
    klasės, ryšiai ir būsenos yra reikalingi būsimoje sistemoje;
  \item yra pagrindas naudotojo sąsajai projektuoti;
  \item yra išeities taškas būsimos sistemos testams projektuoti.
\end{itemize}

Užduočių modeliai nusako visą būsimos sistemos funkcionalumą (sistemos
elgseną, kaip ją mato išoriniai stebėtojai).

Užduočių modeliai naudojami:
\begin{itemize}
  \item sistemos reikalavimams vertinti;
  \item sistemos funkcionalumui aptarinėti (tarp užsakovų, vykdytoj̇ų ir
    kitų suiinteresuotų šalių);
  \item testavimo planui rengti;
  \item sistemos naudotojų instrukcijoms rengti.
\end{itemize}

Jei užduočių modeliai yra rengiami modeliuojant verslo procesus, tai
nagrinėjamoji sistema yra tapatinam su visa tą verslą vykdančia
organizacija, o jei rengiami kuriamai programų sistemai, tai nagrinėjamoji
sistema ir yra ta kuriamoji programų sistema.

% 8 temos 17 skaidrė.

Apie užduotį (taip pat ir kiekvieną jos žingsnį) būtina žinoti:
\begin{itemize}
  \item vykdymo sritį – kokia sistema yra nagrinėjama;
  \item pirminį agentą – kas siekia tikslo;
  \item lygmenį – kokio lygmens yra siekiamas tikslas.
\end{itemize}
Taip pat nagrinėjant užduotį dažnai yra svarbu:
\begin{itemize}
  \item „prieš“ ir „po“ sąlygos – kokios sąlygos turi būti patenkintos
    prieš pradedant vykdyti užduotį ir kokie teiginiai tampa teisingi
    ją įvykdžius;
  \item pagrindinis sėkmės scenarijus – atvejis, kai viskas vyksta
    sėkmingai (nesusidaro jokių ypatingų sistuacijų);
  \item plėtiniai – kokios ypatingos situacijos gali susidaryti 
    vykdant scenarijų.
\end{itemize}

Bet kuri iš užduočių turi būti apibrėžta viename iš šių lygmenų:
\begin{itemize}
  \item sumarinis tikslas – naudotojo tikslų rinkinys;
  \item naudotojo tikslas – savo darbą siekiančio atlikti pirminio agento
    tikslas;
  \item subfunkcija – potikslis arba scenarijaus žingnis, esantis už
    naudotojo tiesioginių interesų ribų (pavyzdžiui, prisijungimas
    prie sistemos).
\end{itemize}

% TODO Suprasti ir įvesti 8 temos 65-66 skaidrės.

\Gls{scenarijaus_sprogimas} – \glsentrydesc{scenarijaus_sprogimas}.
Siekiant išvengti šios situacjos yra naudojamos trys technikos:
\begin{itemize}
  \item použduotys (\emph{«include»} priklausmybė);
  \item plėtiniai (\emph{«extend»} priklausomybė);
  \item variantai.
\end{itemize}
